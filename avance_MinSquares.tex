\documentclass[]{article}
\usepackage{lmodern}
\usepackage{amssymb,amsmath}
\usepackage{ifxetex,ifluatex}
\usepackage{fixltx2e} % provides \textsubscript
\ifnum 0\ifxetex 1\fi\ifluatex 1\fi=0 % if pdftex
  \usepackage[T1]{fontenc}
  \usepackage[utf8]{inputenc}
\else % if luatex or xelatex
  \ifxetex
    \usepackage{mathspec}
  \else
    \usepackage{fontspec}
  \fi
  \defaultfontfeatures{Ligatures=TeX,Scale=MatchLowercase}
\fi
% use upquote if available, for straight quotes in verbatim environments
\IfFileExists{upquote.sty}{\usepackage{upquote}}{}
% use microtype if available
\IfFileExists{microtype.sty}{%
\usepackage{microtype}
\UseMicrotypeSet[protrusion]{basicmath} % disable protrusion for tt fonts
}{}
\usepackage[margin=1in]{geometry}
\usepackage{hyperref}
\hypersetup{unicode=true,
            pdftitle={Implementación de método de optimización convexa con mínimos cuadrados, a través de descenso en gradiente estocástico},
            pdfauthor={Elizabeth Solis; Daniel Sharp; Christian Challu},
            pdfborder={0 0 0},
            breaklinks=true}
\urlstyle{same}  % don't use monospace font for urls
\usepackage{graphicx,grffile}
\makeatletter
\def\maxwidth{\ifdim\Gin@nat@width>\linewidth\linewidth\else\Gin@nat@width\fi}
\def\maxheight{\ifdim\Gin@nat@height>\textheight\textheight\else\Gin@nat@height\fi}
\makeatother
% Scale images if necessary, so that they will not overflow the page
% margins by default, and it is still possible to overwrite the defaults
% using explicit options in \includegraphics[width, height, ...]{}
\setkeys{Gin}{width=\maxwidth,height=\maxheight,keepaspectratio}
\IfFileExists{parskip.sty}{%
\usepackage{parskip}
}{% else
\setlength{\parindent}{0pt}
\setlength{\parskip}{6pt plus 2pt minus 1pt}
}
\setlength{\emergencystretch}{3em}  % prevent overfull lines
\providecommand{\tightlist}{%
  \setlength{\itemsep}{0pt}\setlength{\parskip}{0pt}}
\setcounter{secnumdepth}{0}
% Redefines (sub)paragraphs to behave more like sections
\ifx\paragraph\undefined\else
\let\oldparagraph\paragraph
\renewcommand{\paragraph}[1]{\oldparagraph{#1}\mbox{}}
\fi
\ifx\subparagraph\undefined\else
\let\oldsubparagraph\subparagraph
\renewcommand{\subparagraph}[1]{\oldsubparagraph{#1}\mbox{}}
\fi

%%% Use protect on footnotes to avoid problems with footnotes in titles
\let\rmarkdownfootnote\footnote%
\def\footnote{\protect\rmarkdownfootnote}

%%% Change title format to be more compact
\usepackage{titling}

% Create subtitle command for use in maketitle
\newcommand{\subtitle}[1]{
  \posttitle{
    \begin{center}\large#1\end{center}
    }
}

\setlength{\droptitle}{-2em}
  \title{Implementación de método de optimización convexa con mínimos cuadrados,
a través de descenso en gradiente estocástico}
  \pretitle{\vspace{\droptitle}\centering\huge}
  \posttitle{\par}
  \author{Elizabeth Solis \\ Daniel Sharp \\ Christian Challu}
  \preauthor{\centering\large\emph}
  \postauthor{\par}
  \date{}
  \predate{}\postdate{}


\begin{document}
\maketitle

\newpage

\tableofcontents

\newpage

\section{Introducción}

\subsection{Deficinición de mínimos cuadrados en contexto de optimización convexa}

La optimización convexa consiste básicamente en resolver el problema:\\
\[minimizar \quad f_0(x) \\
  sujeto \ a \quad f_i(x) \leq b_i, \quad i=1,...,m  \\
  donde \ las \ funciones \ f_0,...,f_m:R^n \to R \ son \ convexas\]

Bajo el contexto de optimización convexa, tenemos que podemos resolver
el problema de mínimos cuadrados utilziando el siguiente
planteamiento:\\
\[minimizar \quad f(x):||a^Tx -b||^2_2 \\
sin \ restricciones, \\
con \ A \in R^{m \ x \ n}, m \geq n, a_i^T \ son \ los \ renglones \ de \ A\]

Con esto, sabemos que dado que \[f(x)\] es diferenciable y convexa,
necesitamos encontrar una \[x*\] tal que \[\nabla f(x*) = 0\], que es
condición suficiente para afirmar que ese es el valor óptimo de \[x*\]
que minimiza la función.

A esta definición del problema también se le pueden agregar términos de
regularización, que penalizan la magnitud de los valores de x.

Utilizando estas ideas, se han formulado una serie de algorítmos
iterativos cuyo objetivo es encontrar el valor de la \[x*\] definida
anteriormente.

\newpage

\section{Algorítmos de optimización convexa}

\subsection{Método de Descenso}

El método de descenso consiste en producir una secuencia de valores
\[x^{(k)}, \ k=1,...,\] de la siguiente forma:\\
\[x^{(k+1)}= x^{(k)}+t^{(k)}\triangle x^{(k)}\]\\
donde \[t^{(k)} > 0\] es un escalar denominado como el tamaño de paso.
Esta secuencia cumple la característica de que
\[f(x^{(k+1)})<f(x^{(k)})\], excepto en el caso cuando \[x^{(k)}\] es el
óptimo.

\begin{algorithm}[H]
 \KwData{this text}
 \KwResult{how to write algorithm with \LaTeX2e }
 initialization\;
 \While{not at end of this document}{
  read current\;
  \eIf{understand}{
   go to next section\;
   current section becomes this one\;
   }{
   go back to the beginning of current section\;
  }
 }
 \caption{How to write algorithms}
\end{algorithm}

\subsection{Método de Descenso en Gradiente}

\subsection{Método de Descenso Máximo}

\subsection{Método de Newton}

\newpage

\section{Conclusiones}

\newpage

\section{Anexo}

En esta sección se encuentran los códigos implementados.

\subsection{Código implementado para el método de ordenamiento por inserción}

\newpage

\section{Referencias}


\end{document}
